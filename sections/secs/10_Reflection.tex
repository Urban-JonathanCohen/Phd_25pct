\chapter{Self-Reflection}


It seems that without establishing a set of measurable goals, the New Urban Agenda will be hardly exercised \cite{Caprotti2017, Valencia2018}. Many other institutions ranging from NGOS, businesses, local authorities, or other international agencies also monitor different dimensions of urban life, but still, these indicators capture the output of a phenomena; not the process behind a specific result. Different urban variables are being monitored all the time, (i) in some case the variables may have direct connection to an SDG, such as levels of C02 emotions, (ii) the variable can be a property of the system, such as the total length in km of roads, or (iii) an indirect variable that correlates with the SDG indicator (CO2), for instance; congestion.  \par


Research on cities and urbanity is complex and interactions between different non-linear systems must be taken into account and as a result 
\begin{quote}We cannot predict future cities, but we can invent them. Cities are largely unpredictable because they are complex systems that are more like organisms than machines. Neither the laws of economics nor the laws of mechanics apply; cities are the product of countless individual and collective decisions that do not conform to any grand plan. They are the product of our inventions; they evolve. \parencite{Batty2013}\end{quote} 

After slowly accepting the nature of Planning Problems and the complexity of urbanity, I started to wander what is my role  as a researcher? and How would I like to position myself in the field? (What does it mean to 'do' science in this context?) \par 
In response to both enquiries I decided that as much as, and for as long as possible, I would like to stay away from the Issue Advocate role and mainly stay as a Pure Scientist that when required can contribute to answer specific policy questions by engaging as a Science Arbiter \cite{Brown2008}. \par
One of the advantages, of being at a such an early stage in my research is that, at some point, I can steer the direction of the research and having this in mind I started not to feel so comfortable with building simulations, predictive modelling or trying to solve a concrete urban challenges, such as gentrification. Instead, after much debating with my supervisor we decided we were going to start studying the characteristics of the waste and food systems. By describing parts of the system in greater detail, eventually the links between form, function and performance will be clearer. I believe that if we want to change the specific outputs or performance of a system it is imperative to first disclose its parts and the interaction within its pats and how they relate to other systems.  \par

In this sense my research could be contested at an intellectual plane, discuss to what extent my assumptions are valid ones. The output of the research will hopefully be used by policy makers and contribute to raise awareness of the degree of complexity of this issues. By doing so, I realize that I’m adopting a Precautionary principle, as actions or strategies should be carefully implemented. As wicked as they are, they could backfire and introduce an undesirable effect. 



\section{Ethics concerns around the smart city paradigm}
\subsection{On privacy issues}
Because an important part of my project will deal with exploring and developing new data sets from the ground by using public information and merging it with secondary granular data, at some point I might face some privacy issues.  To start with, there will be copy right issues and limitation on how a data can be handled. The second major data relates to Individual Privacy Rights and by collecting and merging some data systems, some effort should be allocated to debate the legal or ethical implications of these practices.  \par

In order to tackle some of these ethical, with my supervisor we agreed to work with open source software, to open the codes developed during this research project and to free the data created. Under our understanding, working under an open data agenda and collaborating with other institutions contributes to link the research to meet the SDGs. By promoting more transparent processes and interaction with different stakeholders, the science of cities can advance faster and start moving away from the Ivory Tower. \par

This is an assumption, a normative stand we have decided to make as researchers, and I should reflect much more on this issue. For instance, as debated during the lectures, should we open the data sets as soon as the research is done? Or should we wait to individually exhaust the opportunities of the dataset before we release it? Hopefully, when the times comes, the answers to these questions will still be to cooperate, open processes and data; while protecting privacy. \par


\subsection{On technological ubiquity}
Certainly, these new ‘bottom-up’ data sources have contributed to improve our understating of human settlements and in some cases enhance quality of life. As the ubiquity of internet, cellphones, computers, cameras and monitors became a fact we have witnessed the buzz for the Smart City. Academic communities, governments and businesses blinded by the infinite opportunities that mining these new data flow could produce, could not see (or decide to ignore) what George Orwell foresaw in his sci-fiction world. \par

This practice is not new, conscious or intentional; in a completely different context G. B. Shaw reflected on how doctors were assessing other doctor practices and concluded that: 'All professions are conspiracies against the laity' \cite{10.1093/ije/dyg233}. \par

When writing these lines, Shaw wanted to expose that there is a natural tendency of professionals in a same circle to cover themselves and avoid mutual criticisms. This reflection makes me wonder to what extent by using these new gadgets and data sets I am promoting and reinforcing the dynamics of technology ubiquity. If this is the case and I agree with some of the concerns raised against the smart city movement, then doing this research is challenging my beliefs and values of what urbanity should look like. \par

Slowly, as some negative consequences of adopting these consequences are becoming a reality, some academics, activists and institutions have started to rise concern with this new socio-technical regime. \par


%\subsection{On specific products and practices}


\subsection{Closing discussion}
All together, the development of this reflexive essay with the topics covered during the course helped me to raise awareness about my motives to purse a Phd in Urban Analytics. As the course progressed I was able to gain perspective over the context of my research and potential implications to different stakeholders. During these weeks I felt challenged and provoked to question my research and the possible consequences of it. \par 

Besides the concepts around ethics and sustainability, one of the most valuable learning's from the course was that after publication, we as researchers will lose control over the article and how it might be used. Consequently, reflecting on the publication or results prior of publication is important. Try thinking ahead and try to cover potential conflicts.













% \section{Motivation}
% A personal note about why i moved to solve this. 
% Hypothesis.



% It seems that without establishing a set of measurable goals, the New Urban Agenda will be hardly exercised \cite{Caprotti2017, Valencia2018}. Many other institutions ranging from NGOS, businesses, local authorities, or other international agencies also monitor different dimensions of urban life, but still, these indicators capture the output of a phenomena; not the process behind a specific result. Different urban variables are being monitored all the time, (i) in some case the variables may have direct connection to an SDG, such as levels of C02 emotions, (ii) the variable can be a property of the system, such as the total length in km of roads, or (iii) an indirect variable that correlates with the SDG indicator (CO2), for instance; congestion.  \par

% the confluence of these things is my motivation




% \subsubsection{Short summary: Mega-trends and facts}
% •	Iot \par

% ---- The intention to localize and track progress contributes to move in the right direction. However, besides looking at an non-territorial grounded dashboard of indicators, Actions and Strategies are required to revert current unsustainable practices. \par

% The SDG and the subsequent effort to localize the targets are recognized to generate positive attitude and towards the targets. National and Local governments have been showing changes in promoting evidence based policies to validate the commitments of 2015. (PAPER WITH POLITICAL OFFICIAL SAYING HOW MANY KNOW ABOUT) More detailed and granular data is being generated by government to successfully measure the progress towards the SDGs.\par
% %In this context, the Circular Economy paradigm based on the (3)Rs models of sustainability has gained momentum and below it's 'umbrella' different strategies are proposed to correct current unsustainable practices tooted in consumption and production. \par

% Circular Economy paradigm is often associated and focuses on material flow and proposes different strategies \par



% (An Urban Frame of Mind - Alex Krieger in URBAN DESIGN) \par
% Far from coalescing into a singular set of activities, urban design has, over the half century that it gained autonomy from its progenitor design and planning disciplines, evolved less as a technical discipline than as a frame of mind shared by those of several disciplinary foundations committed to cities and to improving urban ways of life. This I consider its strength, though not everyone concurs with such \par

% How can the ideas of circular economy, and more specifically those related to reuse of waste resource be taken into consideration in the planning process?
% How can the circular economy paradigm be incorporated urban and regional planning incorporate the ideas of cir


% Above all this Phd project is steered by the outcomes of the UN Sustainable Development Summit in 2015. The process initiated back in 2013 culminated in the adoption of the 2030 Agenda for Sustainable Development, with specific focus on 17 Sustainability Development Goals (SDGs). \par

% Together with the 17 Goals, a set of sub-goals and 169 performance indicators were agreed upon \parencite{UN2017}. As pointed out above, cities (of all sizes) are found to be important pieces in the sustainable development game. Its relevance has been detailed in the Sustainable Development Goals (i) 11: make cities and human settlements inclusive, safe, resilient and sustainable and (ii) 12: Ensure sustainable consumption and production patterns. \par

% For instance, SDG 11 contains 7 sub-goals (plus 3) and its progress is being monitored by 15 Key Performance Indicators (KPIs). Sub-goal 11.6 commits to reduce the adverse per capita environmental impact of cities, including by paying special attention to air quality and municipal and other waste management by 2030; and its evolution is being tracked by the KPI 11.6.1: The Proportion of urban solid waste regularly collected and with adequate final discharge out of total urban solid waste generated, by cities. The structure of Goal, - Sub goal and KPI(s) is repeated for every SDG and they are self-defined by the UN as a \textit{'blueprint to achieve a better and more sustainable future for all'} \footnote{\url{https://www.un.org/sustainabledevelopment/sustainable-development-goals/}}.\par
