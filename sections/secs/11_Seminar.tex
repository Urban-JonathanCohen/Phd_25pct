\chapter{25\% Seminar Notes}
The 25\% Seminar took place on mid-June 2020 and after the presentation of the project, the space was opened for discussion time. What follows reflects what took place after the presentation.



\section{Round of questions}

\subsubsection{Meta Berghauser Pont}

\begin{enumerate}
    \item Why you defined such a geographical scope? There is some support missing in terms of the geographical scale selection
    
    \item There are some facts missing. This can help to rise a better case, for instance how much  waste is created locally, what type of wastes etc. 
    
    \item Understanding these big numbers and fact can help to understand the problem
    
\end{enumerate}

\subsubsection{Leonardo Rosado}
What is your dream after your Phd? What do you want to achieve as not confident to say it but a GTFS for waste would be nice. This is not the aim of the research but is a necessary step to territorial urban metabolism in city-regions

\subsubsection{Job van Eldijk}
What can be learnt from practitioners? What have you learnt from them?
There are political issues that are out of the scope of this dissertation. But political and institutional factors are extremely relevant to the domain of waste


\subsubsection{Jorge Gil}
How do these systems operate? For instance there is an overlap between instructional-political and administrative boundaries.  Each municipality must take care of its own waste, is this efficient? Environmentally friendly? 


\subsubsection{Leonardo Rosado}
Be aware that also the presence of data can be an indication of political wiliness or concern over environmental or resource issues. 

\subsubsection{Joao Patricio}
Who is your end user? He believes it has to be private, because of the economical benefits. Why a municipality will use it?

\subsubsection{Meta Berghauser Pont}
Be careful. What is your end product. What are you trying to understand? Is it to make a data model? Is to understand spatial relationships?
I still need to be more sharp in this sense. Be more clear about what I want to understand


\subsubsection{Mårten Persson}
Government is changing what waste is collected by cities. Is changing in 2021. Cities will have to manage all types of waste: must collect, hard to do in old towns. Then it must be reused.
Biggest problem for CE: reused resources one more expensive and not as good as virgin resources. Can it be economically viable?
The economic model for CE in Sweden is not applicable in the whole world. Would be extremely complex just in Goteborg. 
\begin{enumerate}
    \item By 2021 it was regulated that in Sweden the municipalities must… take care of waste… this creates problems/challenges of how to do this.
    \item Modelling is going to be impossible. It’s a very complex world
    \item How and where we can install the waste bins?
\end{enumerate}




\subsubsection{Jorge Gil}
Is important for planning to understand how space can be used to contribute CE


\subsubsection{Lars Marcus}
Kill the big dream. Clear emphasis on spatializing things. There is a need to highlight the need to spatialize and how to put things in the territory.
Be clear where the emphasis is, focus now, focus is on the spatial









% MBP: why those delimitation in scope? impact? knowledge gap?
% JC: it’s what spatial can impact most. the two systems (SWM and IS) need the city region

% MBP: Add a better and clearer understanding of the impact and contribution. Add more facts to understand the scale of the problem and impact.

% LR: What is your dream?
% JC: Google maps for waste, GTFS for waste

% Job: stakeholders contribution more than data - the problem, the challenges.

% JG: Geographical space. Organisational space. Information space. We are dealing with the first, but they are related.
% Work can develop compatible/aware to the social, culture and organisational hindrances.

% LR: Mine info from reports in companies, waste collection points.

% JG: The municipality versus the region. Scenarios that break the rules but show the potential.

% JP: not a problem if a private company owns the data. The clients and actors of IS are companies.
% - Who is going to be using the data and what for?
% - User: company, facilitator, Regional body
% - Is there a spatial planning problem?
% JC: To what extent have planning agencies a role in IS?
% JP: Public institutions role is not appening. BRG must be involved.

% MBP: What is your outcome? The tool or the model to understand?

% MARTEN: Bintel collects data.
% Government is changing what waste is collected by cities. Is changing in 2021. Cities will have to manage all types of waste: must collect, hard to do in old towns. Then it must be reused.
% Biggest problem for CE: reused resources one more expensive and not as good as virgin resources. Can it be economically viable?
% The economic model for CE in Sweden is not applicable in the whole world. Would be extremely complex just in Goteborg. 
% (JG: Must talk more with Marten...)

% They don't know what they are doing, so they can't provide or collect data. Less than 1% goes to landfill. 50% is burned instead of reusing.

% Suggest to go for IS or for SWM in terms of modelling.
% IS Just metal and zero waste for example, how to understand how it can be climate compensated. Complex systems thinking.
% SWM, easy to access the data. They can provide data from 5000 households.
% Realising CE is a challenge. Model might work, but only for 1 waste.

% LR: JC is not doing the complete dimensions (material, social, economic) but a spatial dimension of the system.
% CE at the start of the presentation, but then is only reuse of resources, and in the end it's just SWM, a very small part of CE.
% CE interesting but too complex can stop from focusing on details.

% Marten: KPI how expensive can the collection of waste be. Where to place bins?
%                 How far are people wanting to go? what happens with traffic? How much does it cost?

% LM: 
